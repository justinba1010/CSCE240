\documentclass[fleqn]{article}
\usepackage[left=1in, right=1in, top=1in, bottom=1in]{geometry}
\usepackage{mathexam}
\usepackage{verbatim}

\ExamClass{CSCE 240}
\ExamName{Homework 2}
\ExamHead{Due: 19 February 2019}

\let
\ds
\displaystyle

\begin{document}
\ExamInstrBox {
  You shall submit a zipped, \textbf{and only zipped}, archive of your homework
  directory, hw2. The directory shall contain, at a minimum, the files
  \texttt{parse\_scores.cc} and \texttt{parse\_scores.h}. Name the archive
  submission file hw2.zip \\

  I will use my own makefile to compile and link to your
  \texttt{parse\_scores.cc} and \texttt{parse\_scores.h} files. You must submit,
  at least, these two files.
}
%
I decided that Python was just too slow and so not wanting to deal with
JSON in C++, I want you to write a library of C++ functions. \\
%
\\
The functions shall accept arrays of strings which I will provide to your
functions. I will ensure that the arrays are of no less than the indicated size.
\\
%
\\
The format remains the same: number\_of\_students student\_id number\_of\_grades
grade\_0 grade\_1 grade\_n \\
\\
Be mindful. Sometimes the data gets corrupted and partially dropped. Though I
can assure you the array size is correct, the format may indicate more students
or grades than the array actually holds. I am not sure what is causing this bug,
so you must alert me when corruptions of this sort are detected so that I may
respond accordingly. \\
%
\\
Read the provided header file documentation for instructions on how the
functions should work. \\
%
\\
I have provided you a basic test app which you can use to ensure that your code
is somewhat passing. I would suggest a much more rigorous testing scheme. Your
code must behave as indicated in the documentation. Passing the tests is not
enough to get full credit. \\
%
\\
Late assignments will lose 25\% per day late, with no assignment begin accepted
after 3 days (at 100\% reduction in points).\\
\\
%
Check your syllabus for the breakdown of grading.
\end{document}

