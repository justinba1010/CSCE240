\documentclass[fleqn]{article}
\usepackage[left=1in, right=1in, top=1in, bottom=1in]{geometry}
\usepackage{mathexam}
\usepackage{verbatim}

\ExamClass{CSCE 240}
\ExamName{Homework 7}
\ExamHead{Due: 03 May 2019}

\let
\ds
\displaystyle

\begin{document}
\ExamInstrBox {
  You shall submit a zipped, \textbf{and only zipped}, archive of your homework
  directory, hw7. The directory shall contain, at a minimum, the files and
  directories \texttt{src/matrix.cc} and \texttt{inc/matrix.h}. Name the archive
  submission file hw7.zip \\

  I will use my own makefile to compile and link to your \texttt{src/matrix.cc}
  and \texttt{inc/matrix.h} files. If you do not submit in this way \textbf{YOU
  WILL RECEIVE A ZERO}.
}
%
This assignment tests your ability to manage memory of a generic type. You are
to implement the basics of a matrix class, using a two-dimensional array of
templated type as your representation. You may assume the generic type will have
all the same operators defined as any numeric type (+, -, *, /, =, etc...). \\
%
\\
I will be using white-box testing, so the friend class MatrixTester along with
its forward decl must remain in place and unchanged. \\
%
\\
Read the provided header file documentation for instructions on method
functionality. They are currently defined for a \texttt{double} type. That must
be changed to a template type. \textbf{YOU} must read the documentation and
decide which data types need to be modified to generic. \\
%
\\
I have provided you a set of test apps which you can use to ensure that your
code is, at least partially, correct. \textbf{I would suggest a more rigorous
testing scheme, especially testing your assignment operator}. The tests use a
double as the template for the matrix. I \textbf{WILL} use other types. I would
recommend ensure that it works, at least, for an integer type and maybe an
unsigned integer type as well. \\
%
\\
There will be no late assignments accepted. I am giving you until the last
possible time to get it in. I repeat, I \textbf{WILL NOT} accept late
assignments. \\
\\
%
The point allocation is specified in the header file. There will be no
points for compilation that does not pass at least one test. Style points will
be all-or-nothing for this assignment. \\
\\
%
If you accumulate the points for this assignment, you will see that there are 10
points to earn, but the assignment counts as 5. This means that there is up to
5 bonus points available. \\
\\
\end{document}

