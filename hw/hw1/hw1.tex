\documentclass[fleqn]{article}
\usepackage[left=1in, right=1in, top=1in, bottom=1in]{geometry}
\usepackage{mathexam}
\usepackage{verbatim}

\ExamClass{CSCE 240}
\ExamName{Homework 1}
\ExamHead{Due: 01 February 2019}

\let
\ds
\displaystyle

\begin{document}
\ExamInstrBox {
  You shall submit a zipped, \textbf{and only zipped}, archive of your homework
  directory, hw1. The directory shall contain, at a minimum, the file
  \texttt{parse\_score.cc}. \\

  I will use my own makefile to make your \texttt{parse\_score.cc} file. Do not
  use a header for this assignment. My grader will not look for one.
}
%
I have a large amount of text data with student scores stored as series of
integers and floating point values. I would like to read that data in JSON
format. \\
%
\\
Given the format: number\_of\_students student\_id number\_of\_grades grade\_0
grade\_1 grade\_n \\
%
\\
Read in each of the number\_of\_students students and write their id and
number\_of\_grades grades to the standard output stream as indicated in the provided source code.
%
\vspace{1.0em} \\
The application should perform as follows:
\begin{itemize}
  \item The program is called and and is run with no prompt for input: \\
  \\
  ./parse\_source

  \item The program reads from the standard input stream a series of characters
  in the indicated format e.g. \\
  \\
  2 1234 3 98.7 87.92 77.32 2345 3 93.1 90.23 81.21 \\
  \\
  and parses the values as indicated in the given source file:
  \begin{verbatim}
{
  "students" : [
    { "id" : 1234, "grades" : [ 98.7, 87.92, 77.32 ] },
    { "id" : 2345, "grades" : [ 93.1, 90.23, 81.21 ] }
  ],
  "max_id" : 1234,
  "max_score" : 98.7
}  
  \end{verbatim}

  \item I have provided you a basic test app which you can use to ensure that
  your code is somewhat passing. I do not test the individual values of ``id''
  and ``grade'', but I will in my full tester. This means that, while you could
  only print the ``max\_id'' and ``max\_score'' to pass the provided tester, you
  will not pass my full tester.

  \item If you are having problems getting valid JSON, copy the output from your
  program and try a JSON validator such as: https://jsonformatter.curiousconcept.com/

\end{itemize}
\vspace{1.0em}
%
I have provided you source for parse\_scores app and a make file. In addition, I
have given you a python script to test your app. You should definitely read the
makefile and I would encourage you to read the python. It is simple code and an
easy intro to python.\\
\\
%
Late assignments will lose 25\% per day late, with no assignment begin accepted
after 4 days (100\% reduction in points).\\
\\
%
Check your syllabus for the breakdown of grading.
\end{document}

